\documentclass[tikz,dvisvgm]{standalone}
\usetikzlibrary{decorations.markings}
\usetikzlibrary{decorations.text}
\usetikzlibrary{shapes,arrows,backgrounds,fit}

\tikzstyle{block} = [draw, fill=white, rectangle, minimum height=3em, minimum width=6em]
\tikzstyle{blocksq} = [draw, fill=white, rectangle, minimum height=3em, minimum width=3em]
\tikzstyle{gain} = [draw, fill=white, isosceles triangle]
\tikzstyle{sum} = [draw, fill=white, circle]

\begin{document}
	\begin{tikzpicture}[node distance=1.5cm, >=latex']
		\node[coordinate] (input) at (0, 0) {};
        \node[gain, right of=input, node distance=1cm] (ctlgain) {$G$};
		\node[sum, right of=ctlgain] (sub) {};
		\node[block, right of=sub, node distance=2cm] (stateequ) {$\dot x = Ax + Bu$};
        \node[coordinate, right of=stateequ] (feedback) {};
        \node[block, right of=feedback] (outputequ) {$y = Cx$};
        \node[coordinate, right of=outputequ] (output) {};
        \node[gain, below of=stateequ, shape border rotate=180] (ctlfeedback){$K$};

        \draw[->] (input) -- node [above] {$r$} (ctlgain);
        \draw[->] (ctlgain) -- (sub);
        \draw[->] (sub) -- node [above] {$u$} (stateequ);
        \draw[-] (stateequ) -- (feedback);
        \draw[->] (feedback) -- node[above] {$x$} (outputequ);
        \draw[->] (outputequ) -- node[above] {$y$} (output);
        \draw[->] (feedback) |- (ctlfeedback);
        \draw[->] (ctlfeedback) -| node[pos=0.99, below left] {$-$} (sub);

        \node[coordinate] (input2) at (1.5, -5) {};
        \node[sum, right of=input2, node distance=1cm] (error2) {};
        \node[block, right of=error2, minimum width=3em] (preint2) {$\int$};
        \node[gain, right of=preint2] (errorgain2) {$F_2$};
        \node[sum, right of=errorgain2] (sub2) {};
        \node[block, right of=sub2, node distance=2cm] (stateequ2) {$\dot x = Ax + Bu$};
        \node[coordinate, right of=stateequ2] (feedback2) {};
        \node[block, right of=feedback2] (outputequ2) {$y = Cx$};
        \node[coordinate, right of=outputequ2] (output2) {};
        \node[coordinate, right of=output2, node distance=1cm] (output3) {};
        \node[gain, below of=stateequ2, shape border rotate=180] (stategain2) {$F_1$};
        \node[coordinate, below of=stategain2] (waypoint2) {};

        \draw[->] (input2) -- node [above] {$r$} (error2);
        \draw[->] (error2) -- (preint2);
        \draw[->] (preint2) -- node [above] {$z$}(errorgain2);
        \draw[->] (errorgain2) -- node [pos=0.99, above left] {$-$}(sub2);
        \draw[->] (sub2) -- node [above] {$u$} (stateequ2);
        \draw[-] (stateequ2) -- (feedback2);
        \draw[->] (feedback2) -- node [above] {$x$} (outputequ2);
        \draw[-] (outputequ2) -- (output2);
        \draw[->] (output2) -- node [above] {$y$} (output3);

        % State feedback
        \draw[->] (feedback2) |- (stategain2);
        \draw[->] (stategain2) -| node [pos=0.99, below left] {$-$}(sub2);

        % Error feedback
        \draw[->] (output2) |- (waypoint2) -| node [pos=0.99, below left] {$-$} (error2);

        \begin{scope}[node distance=1.2cm]
            \node[coordinate, right of=output, node distance=2cm] (input3) {};
            \node[blocksq, right of=input3] (inputmat) {$B$};
            \node[sum, right of=inputmat, node distance=1.2cm] (sumstate) {};
            \node[blocksq, right of=sumstate] (intstate) {$\int$};
            \node[coordinate, right of=intstate] (feedbackstate) {};
            \node[blocksq, right of=feedbackstate] (outputmat) {$C$};
            \node[coordinate, right of=outputmat] (outputstate) {};
            \node[blocksq, below of=intstate] (statemat) {$A$};
            \node[coordinate, above of=feedbackstate, node distance = 0.5cm] (outputstate2) {};

            \node[block, below of=statemat, align=left] (fullequ) {$\dot x = Ax+Bu$\\$y = Cx$};

            \draw[->] (input3) -- node[above] {$u$} (inputmat);
            \draw[->] (inputmat) -- (sumstate);
            \draw[->] (sumstate) -- (intstate);
            \draw[-] (intstate) -- (feedbackstate);
            \draw[->] (feedbackstate) -- (outputmat);
            \draw[->] (outputmat) -- node[at end, right] {$y$} (outputstate);

            \draw[->] (feedbackstate) |- (statemat);
            \draw[->] (statemat) -| (sumstate);

            \draw[->] (feedbackstate) -- node[at end, above] {$x$} (outputstate2);

        \end{scope}

        \begin{scope}[on background layer]
            \node[draw,
                densely dotted,
                inner sep=0.5cm,
                fit= (input3) (statemat) (outputstate) (outputstate2)
            ] {};
        \end{scope}


	\end{tikzpicture}
\end{document}